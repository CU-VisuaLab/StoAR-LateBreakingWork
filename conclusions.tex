\section{Discussion}

Our findings gathered throughout the design process of StoAR led us to a few conclusions about the potential for use of Augmented Reality in the retail environment.

\subsubsection{Merging Experiences}
We found that augmented reality merges the online and in-store shopping experiences in a way that does not negate aspects of in-store shopping that people enjoy, while adding aspects from the online experience which shoppers prefer to the in-store one.  Based on our early survey on users' shopping tendencies, users see price and review score as being most important in making purchasing decisions.  We see these desires reflected in the users' views of our system at early and final prototyping stages.

Participants in our low fidelity prototype study preferred sparse bits of context-aware content.  Based on feedback from the final questionnaire, users envision systems such as ours affording them quick review and product information.  Users also envision an augmented reality system such as the one presented used for comparing products while shopping.

\subsection{Unique Shopping Experience}
A few key findings lead us to the conclusion that augmented reality provides a unique shopping experience.  
Our final prototype provided users with static content overlaid onto a store environment.  Users responded that they would expect interaction from a system such as this one.  Interactive pervasive content distinguishes an augmented reality system from a static sheet of paper displaying relevant information.

Data collected from the final questionnaire also points toward augmented reality providing users visualizations of the product used in its intended environment and/or use cases, particularly room design and game previews.  These visualizations and the ability to virtually unbox a product further empower the user in making purchasing decisions.
Considering the final questionnaire's finding about quick bits of information, augmented reality provides somewhat of a middle-ground between an on line and an in-store shopping experience.  Users want to be empowered with pertinent information, but it must be easily accessible and digestible--not interfering with the immediacy of the in-store experience.

\subsection{Limitations \& Future Work}
Our prototyping approach was to create a store environment to navigate complemented with review content in virtual reality.  This prototype represents a physical store with augmented reality content.  Several users provided feedback indicating that they did not understand the concept of prototyping augmented reality in virtual reality.  In some cases, users gave feedback perceiving the system as a virtual store to virtually navigate.  More effective means of prototyping augmented reality, ideally using the system's intended hardware, would eliminate some of these misconceptions.

Users also expressed some concern about stumbling as a result of the added distractions in the environment.  In some cases, these concerns may have stemmed from a lack of understanding, as users may have thought their entire view of their surroundings would be blocked by a virtual reality head-mounted display.  Even if not a misconception, this finding points toward use of the intended hardware in prototyping for augmented reality.  If we used an augmented reality head-mounted display, we could test for safety in addition to usability.

\todo[inline]{Future work: Comparison of delivery method: mobile phones vs. HMDs}

\section{Conclusion}
\todo[inline]{write this :-)}
We did a thing, it was interesting, we learned from it.