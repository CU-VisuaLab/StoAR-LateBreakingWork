\section{Discussion}
% First, revisit the rq and findings
We explored how understanding the trade-offs of parallel physical and virtual experiences could inform mixed reality applications in the context of retail shopping. Our findings suggested that participants appreciated the ability of mixed reality experiences to offer more rapid access to information about products while preserving the physical interactions offered by in-store experiences. Using a design-based methodology allowed participants to envision ways that AR technologies might enable new experiences beyond those offered by physical or virtual methods alone, such as mediating evaluative interactions with products like guided demos or simulated placement.  
% Then, introduce the synthesis
Our findings, gathered throughout the design process, highlight the potential for AR in retail environments and open new avenues for understanding how we might effectively derive new AR experiences from existing processes. \todo{revise the wording on this last bit}

\subsection{Merging Physical \& Virtual Experiences}
We found preliminary evidence that AR can merge online and in-store experiences to benefit the consumer by preserving aspects of in-store shopping that people enjoy while adding preferred aspects of online experiences.  In the initial survey, participants cited the ability to find the best price and access reviews as the most important reasons to use technology in shopping. However, the participants who interacted with the StoAR prototypes indicated that it would allow them to better analyze product specs \emph{in situ}, compare reviews and ratings against their own experiences, and mediate their interactions with physical products through virtual demos.
%While StoAR focused on allowing consumers access to virtual data while in a physical store, some participants also envisioned a role for fully-immersive shopping experiences based on the prototypes, suggesting that such technologies might allow them to ``look at products without being at a brick and mortar store,'' prioritizing the convenience of online shopping. 

%Based on our early survey on users' shopping tendencies, users see price and review score as being most important in making purchasing decisions.  We see these desires reflected in the users' views of our system at early and final prototyping stages.
% Users also envision an augmented reality system such as the one presented used for comparing products while shopping. 
These findings align well with prior studies in e-commerce, which demonstrate the utility of price and product comparison in online shopping \todo{cite source(s) MW: I believe we did in the Intro}; however, our findings also show that blending the benefits of online shopping with traditional retail outlets may provide valuable experiences not accessible online. \todo{example? MW: Not sure what we're getting at here.  Like that people can touch and interact with products physically like they already can in a physical store?} Our design-based approach confirmed important aspects of online shopping shared by participants in the initial survey. We also found that people's perceptions of how AR might benefit their decision making evolved as the prototypes grew more sophisticated. These shifting perceptions suggest that using prototypes as design prompts is beneficial for helping people envision how AR experiences might differ from conventional methods. 

\subsection{New Opportunities for Blended Experiences}\todo{better section title}
We also found evidence that AR might enable unique kinds of decision support that neither physical nor virtual experiences alone can provide. For example, our high fidelity prototype provided static summaries of the product data identified as critical in the low fidelity study, similar to those accessible through a product specification sheet. However, in the high fidelity prototype, participants expressed an additional desire to interact with the system, such as the ability to virtually ``pin'' relevant information for easy access as they navigated the store or access to specific details on demand. 

Participants also reported a desire to use AR to simulate their own at-home context or use case for a particular product, particularly room design and game previews. They felt such features would allow them to ``look at products without being at a brick and mortar store,'' prioritizing the convenience of online shopping. The ability to explore these simulations to virtually unbox a product would provide additional information inaccessible in traditional experiences.

These findings collectively suggest that designing mixed reality applications is not as simple as blending the best aspects of both the virtual and physical experiences. AR technologies allow new methods for supporting decision making not afforded by purely physical or virtual methods. Instead, designers must critically reflect on how AR may effectively mediate novel kinds of interactions to transcend traditional approaches and provide consumers with new forms of decision support. Future work can explore how the capabilities of these technologies and properties of the target domain and task might inform novel AR experiences.
%Not as simple as drag and drop: need for understanding at a glance communication to provide effective delivery.  

\subsection{Designing for Effective Communication}
In our final questionnaire, participants frequently commented on the need to carefully limit the amount of information provided by the interface. Participants want to be empowered with pertinent information, but it must be easily accessible and digestible to avoid interfering with the immediacy of the in-store experience. They expressed some concern about trying to process too much information and about the display being too distracting, inhibiting their ability to navigate the physical store. Instead, participants preferred sparsely presented information in-context, allowing them to access relevant review and product information at a glance. 

Future systems will benefit from a better understanding of the balance between information presented and visual space consumed. Our findings identify a need to understand how AR systems might balance communicative power with interaction to deliver necessary information at a glance. A lack of consensus amongst participants as to what information is ``necessary'' suggests opportunities for designing intelligent interfaces and customized experiences not available in real world environments in order to support individual decision making. 
%modes for interaction, interface design, customization? Also, balance needs of the consumer and producer: how can we empower users while working within the constraints of the available params

\subsection{Limitations \& Future Work}
Our high fidelity prototype used a simulated store display rather than a real world environment.  While this approach allowed us to conduct a preliminary evaluation of the prototype application without the added complexity of instrumentation or physical obstacles, it also limits our ability to fully characterize the affordances of our prototype in practice. Previous work such as et al. \cite{macintyre2004dart} deals with effectively prototyping augmented reality experiences. Prototyping in the intended medium, in our case an augmented reality head-mounted display, yields a more rapid prototyping process than a separate device or framework. 
%DS-Removed this for space
%Some participants asked for verbal clarification as to whether to treat the simulated environment as a proxy for the physical environment. This confusion provides preliminary evidence of the challenges of prototyping AR applications using VR simulations: it can be difficult for participants to disentangle perceptions of the AR components from that of the fully immersive simulation. 
While we believe that the design guidance provided by our work can inform effective blended experiences, we also realize that further refining and implementing StoAR in full AR is a critical next step. 
%Even if not a misconception, this finding points toward use of the intended hardware in prototyping for augmented reality.  If we used an augmented reality head-mounted display, we could test for safety in addition to usability.
%This prototype represents a physical store with augmented reality content.  Several users provided feedback indicating that they did not understand the concept of prototyping augmented reality in virtual reality.  In some cases, users gave feedback perceiving the system as a virtual store to virtually navigate.  
%More effective means of prototyping augmented reality, ideally using the system's intended hardware, would eliminate some of these misconceptions. \todo{frame this in terms of trade-offs: vr gives nearly full control, but removes many of the complexities of experiencing these systems in practice. Here, it allows us to gain a sense of how people would respond to the system alone, but future work is needed to understand how these responses generalize to the noisy contexts of stores}  In some cases, these concerns may have stemmed from a lack of understanding, as users may have thought their entire view of their surroundings would be blocked by a virtual reality head-mounted display.  

In our first survey, many participants reported that they use their mobile devices to engage in online shopping while in a physical store. While this approach allows for people to retrieve in-depth information \emph{in situ}, it also generally requires significant effort to locate and compare relevant information on the fly. Participants reported that they ``like the idea of hands-free/ambient information'' offered by HMDs over mobile devices, but were also concerned about the legibility of the information presented in AR. Our future work will directly compare traditional mobile devices to the StoAR approach to better understand the trade-offs of blended and parallel methods for decision making. 

\section{Conclusion}
\todo[inline]{write this :-)}
We did a thing, it was interesting, we learned from it.