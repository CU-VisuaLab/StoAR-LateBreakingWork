Many tasks can be completed through either real world or virtual experiences; however, each experience has its own unique set of trade-offs.\todo{this first sentence needs to be reworked} In this work, we explore how designers might employ mixed reality to effectively leverage the respective benefits of real world and virtual experiences, using retail shopping as a case study. Through a series of surveys, prototypes, and design evaluations, we identify primary trade-offs for in-store and online shopping experiences and ways in which augmented reality might enhance user decision making in retail environments. 
Our findings suggest that users perceive AR as an effective means for communicating contextualized, at-a-glance access to critical information, such as price comparisons and review information, to support their decision making process while retaining the convenience and immediacy of physical interactions with products.  These studies additionally provide preliminary evidence that AR may offer opportunities to transcend traditional retail experiences by offering new physical/virtual blended interactions only accessible through immersive technologies.
%As much of store commerce has migrated online, consumers are growing to appreciate aspects of online shopping--in several ways more than the in-store experience.  Using the appropriate technology, one could effectively merge aspects of the in-store and online shopping experiences.  
%We present StoAR, a system that employs augmented reality to deliver an important subset of the online shopping experience without sacrificing aspects of the in-store experience which customers appreciate.  We have provided users with  
% We administered surveys and user tests throughout the design process.  We conclude that augmented reality is an effective vehicle to merge online and in-store shopping experiences, providing a novel means of customer engagement.
