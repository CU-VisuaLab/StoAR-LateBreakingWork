Augmented reality (AR) systems allow designers to blend aspects of physical and virtual experiences. Previous work discusses adding new functionality to an exisiting experience, but provides little guidance for effectively merging parallel experiences. We use retail shopping as a case study to understand how user-centered design practices might enable designers to craft novel AR applications from existing parallel in-store and online experiences. Through a series of surveys, prototypes, and design evaluations, we work directly with target users to identify trade-offs of in-store and online shopping and derive a set of design considerations for how augmented reality might support consumer decision making in traditional retail environments. Our findings suggest that users perceive AR as an effective means for contextualized, at-a-glance access to critical information such as price comparisons and reviews, while retaining the convenience and immediacy of in-store shopping. We also found preliminary evidence of how blending existing parallel experiences might inspire novel immersive interactions that transcend traditional retail experiences.
%did some small edits and cuts so it's 153 words. guidelines say 150 -EH
% And I did another little bit of pruning to bring it to 150 - JRB
%As much of store commerce has migrated online, consumers are growing to appreciate aspects of online shopping--in several ways more than the in-store experience.  Using the appropriate technology, one could effectively merge aspects of the in-store and online shopping experiences.  
%We present StoAR, a system that employs augmented reality to deliver an important subset of the online shopping experience without sacrificing aspects of the in-store experience which customers appreciate.  We have provided users with  
% We administered surveys and user tests throughout the design process.  We conclude that augmented reality is an effective vehicle to merge online and in-store shopping experiences, providing a novel means of customer engagement.
