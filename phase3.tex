\section{Phase 3: High Fidelity Prototype}

\subsection{Methodology}
We used the 
insights generated
in Phase Two to design a fully immersive StoAR prototype for use in head-mounted displays. Because we did not have access to a retail testing environment, we simulated a retail display in virtual reality based on configurations found in a local retail outlet. We then added an augmented reality interface to the simulation to provide review and price information (Fig. \ref{figures:HiFiScreenshots}).\todo{briefly describe the important pieces informed by the low-fi study} While our use of a VR store simulation removes immediate access to real life product found in a real world store, the fully immersive environment allowed us to closely control the relationship between the prototype AR interfaces and simulated products. %and reduced the environmental complexity to allow participants to focus solely on the prototype implementation. 

%\begin{figure}
%	\includegraphics[width=0.9\columnwidth]{figures/3Panel}
%	\includegraphics[width=0.9\columnwidth]{figures/2Panel}
%	\includegraphics[width=0.9\columnwidth]{figures/DetailReviews}
%	\caption{Screenshots of high fidelity protoype VR experience}
%	\label{figures:HiFiScreenshots}
%\end{figure}

We recruited 20 participants from a local research expo to evaluate our prototype. Participants first freely navigated the virtual representation of a store augmented with static content containing information about each device.
%Given space constraints, the track pad on one of the Vive's controllers was employed for users to navigate within the scene.
After navigating the scene, participants completed a short survey that asked for their perceptions of the prototype's usability, potential impact on decision-making, utility of individual design components, perceived trade-offs compared to existing technologies, and potential limitations of the approach. They also provided feedback on additional applications where they envisioned using mixed reality experiences for decision making. As with the initial online survey, we analyzed the qualitative feedback provided by users by clustering their responses into common themes.
%The survey asked for specific product-domains to which our system could be applied (i.e. electronics, furniture, books).  

\begin{marginfigure}
	\begin{minipage}{\marginparwidth}
		\centering
		\subfloat[][]{\includegraphics[width=0.9\columnwidth]{figures/3Panel}}
		\vfill
		\subfloat[][]{\includegraphics[width=0.9\columnwidth]{figures/2Panel}}
		\vfill
		\subfloat[][]{\includegraphics[width=0.9\columnwidth]{figures/DetailReviews}}
		\caption{In Phase Three, we designed an immersive simulated retail environment augmented with supplemental AR menus. We then measured participant responses to this StoAR prototype to provide preliminary data about what aspects of an augmented shopping experience might best support consumer decision making. \textbf{DNS: Please revise this wording :-)} }\label{figures:HiFiScreenshots}
	\end{minipage}
\end{marginfigure}

Participants said they envision using this platform for comparison shopping, quickly seeing reviews, prices, and product specifications, and visual demonstrations of product use. 
Lower depth of product information was provided as a trade-off of the system. \todo{JRB: Clarification please.}
Participants expressed concern about digital content distracting them from the physical environment. 
%Having presented them with laptops, we asked participants for what other products they 
%think % JRB: w/c? Thought?
%augmented reality could aid them in decision-making, with the results shown in figure . \todo{MW: Games, toys, furniture, and clothing.  Honestly, this finding doesn't really line up with anything else, unless we start drawing conclusions about "They said furniture because of potential for visualizations" or something like that. We have a figure if we want to keep this point.} 
%DNS: This is talked about in the previous and next paragraph, so removed for space and to focus this paragraph on findings. Feel free to revise there if you'd like, but I think mentioning it is enough here. 

We prompted participants for additional features they would like to have in a retail augmented reality system. Participants detailed several desired interactions with the provided content. In line with feedback from the low-fidelity prototypes, participants wanted to keep information ``pinned'' within a single view such that they could comparison shop, rather than needing to physically navigate between products, as implemented in our prototype. Participants also wanted to compare prices of a product against other stores' prices. \todo[inline]{JRB: Same comment as Phase 2. This is a list of what they told you. Can you take it one step further and tell us what this means? What are the implications of these comments? You said you did this -- "We explored how understanding the trade-offs of parallel physical and virtual experiences could inform mixed reality applications in the context of retail shopping." -- so tell me what the results of this phase told you about those trade-offs. DNS--Yes, please! Especially look how things evolved from the first phase until here. Also, for Phase 2 \& 3, the idea of new applications, such as Demos came up. Those should be brought up in the findings as evidence of consumer thinking evolving with the design prompts--a big win for this type of methodology.}
%If presented with the option, participants said they would consider purchasing from another source. 