\section{Results}
\todo[inline]{JRB: Structural comment: This paper might flow more easily if instead of having a "Methods" section and then a "Findings Section", you go with three sections -- one for each phase. You would start the section with all the methods information, and the end it with the findings. Because the findings should directly inform the next phase of the study, this might streamline the paper. DS, thoughts?}
\todo{we found x, which is supported by [theoretical finding about AR in some paper}

\subsubsection{Online Survey}
We collected responses from 78 participants over social media. 

\begin{itemize}
\item Participants appreciate the immediacy and physical interaction with products in a store.
\item Participants dislike about shopping in-store:
	\begin{itemize} \compresslist%
		\item Lack of ability to comparison shop in-store.
		\item Not getting the lowest price or feeling like they are paying more.
		\item Staff trying to influence purchase decisions.
	\end{itemize}
\item Participants appreciate about online shopping:
	\begin {itemize} \compresslist%
		\item Convenience and efficiency.
		\item Seeing reviews and comparison shopping.
	\end{itemize}
\item Participants dislike the shipping charges and subsequent wait times.
\item When asked for their most important factors in making shopping decisions, 96\% of users said "Price" and 76\% said "Reviews" were in their top three factors out of six options. \todo{add qualitative analysis about how we expect x number of people to select each factor on average and these are the only two that surpassed that number}
\item Participants spend more time researching and comparing more expensive products than cheaper ones.
\end{itemize}
\todo{figure summarizing important results here}

\subsection{Low Fidelity Prototype}
We recruited XXX participants for this study. \todo{How many subjects and where from?}
\begin{itemize}
	\item Participants expressed a desire to view specifications of two different laptops in the same view.
	\item Participants were more receptive to quick and less information than to a fuller, menu-based approach.
	\item Participants mentioned a desire to toggle display of content.
	\item A participant explained product demos as a useful application of augmented reality in retail.
\end{itemize}

\subsection{High Fidelity Prototype}
We recruited XXX participants from a local research expo to complete this study. \todo{how many subjects?}
\begin{itemize}
	\item How participants envision use of this platform:
	\begin{itemize} \compresslist%
		\item Comparison shopping.
		\item Quickly see reviews, prices, and product specifications.
		\item Video or visual demonstrations of how the product is used.
	\end{itemize}
	\item Tradeoffs of this system:
		\begin{itemize} \compresslist%
			\item Less depth in the product information listed.
		\end{itemize}
	\item Other useful features in augmented reality retail:
	\begin{itemize} \compresslist%
		\item Several different expected interactions with the static content.
		\item Keeping information within view to comparison shop.
		\item To use this system to compare competitors' prices and would consider buying from another retailer if presented with a better price.
	\end{itemize}
	\item Concerns about the system:
	\begin{itemize} \compresslist%
		\item Participants expressed concern with digital content being distracting from the physical environment.
	\end{itemize}
	\item Other types of products augmented reality could be used for: games, toys, furniture, and clothing.
\end{itemize}