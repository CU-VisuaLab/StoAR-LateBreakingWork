\section{Methodology}

We explored how mixed reality might leverage physical and virtual shopping experiences through a combination of open-ended surveys and responses to prototype StoAR platforms. Our surveys primarily focused on differences in physical and virtual experiences for decision making (e.g., the choice to purchase a certain product). 

\subsubsection{Online Survey}
We first conducted a preliminary survey to identify important aspects of customers' in-store and online shopping experiences. We asked participants
% to incorporate, we first sent out a survey via social media.  We asked users 
to identify aspects of in-store and online shopping experiences that they preferred and about the ways they currently use technology in shopping. \todo{does this feel like a reasonable synopsis?}  We also gathered information about critical factors when deciding whether to purchase a product.  
%We gathered data on shoppers' propensity to use a phone or tablet when shopping and to research a product before making a purchase decision. 

We collected responses from 78 participants over social media. We used the responses from our online survey to isolate factors of in-store and online shopping that were most important to our participants.  \todo[inline]{Can you add some details about how you analyzed this data here? Maybe replace the last sentence with something more specific about the outcomes.} 

\subsection{Low Fidelity Prototype}
Our preliminary survey pointed to the utility of augmented reality for merging both in-store and online experiences. Several factors identified as important for in-store decision making, such as being ``hands-on with the product'' and leaving the store with the product, are not readily feasible in an online experience; however, critical factors of online experiences, such as access to reviews and ease of comparison, can be brought into the store environment. We hypothesized that an augmented reality application that supplemented a traditional in-store experience with immediate access to core aspects of online shopping would improve consumer's confidence in their purchasing decisions. \todo{replace the last bit of this sentence with what was actually tested}

We tested our hypothesis using a paper prototype as a design prompt. This prototype consisted of online content drawn onto transparency sheets and overlaid onto an image of a Best Buy store. We used the low fidelity prootype to conduct an A/B test of a menu-based, hierarchically structured user experience, and a context-aware virtual overlays of product information. \todo{any specific information}  We instructed participants to use the prototyped system to aid in making a laptop purchase decision. \todo{what specific AR components were tested here (e.g., reviews, product comparisons, specs, etc.)? What were your measures here (e.g., confidence in the decision, time to decision, etc.)}
\todo{figure of paper prototype}

\subsection{High Fidelity Prototype}
We used the responses to design a fully immersive StoAR prototype for use in head-mounted displays. Because we did not have access to a retail testing environment, we designed a mock laptop display in virtual reality based on configurations found in a local retail outlet. Our augmented reality interface provided XXX.\todo{briefly describe the important pieces informed by the low-fi study} In contrast to a real store setting, the fully immersive environment allowed us to closely control the relationship between the prototype augmented reality interface and simulated products and displays while also reducing environmental complexity to allow participants to focus solely on the prototype implementation. 

We conducted an in-person study to measure people's responses to the immersive prototype. Participants freely navigated the virtual representation of a store augmented with static content containing information about each device. 
%Given space constraints, the track pad on one of the Vive's controllers was employed for users to navigate within the scene.
After navigating the scene, participants completed a survey about the prototype's usability and their perceptions of augmented reality for an in-store shopping experience, including applications where participants anticipated a mixed reality experience might be particularly beneficial.  
%The survey asked for specific product-domains to which our system could be applied (i.e. electronics, furniture, books).  
The survey also measured subjective impact on decision-making, individual design components, perceived trade-offs compared to existing technologies and potential limitations of the approach.
