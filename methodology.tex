\section{Methodology}

We explored how AR might leverage insights from existing physical and virtual shopping experiences through a combination of open-ended surveys and user responses to prototype StoAR platforms. We primarily focused on understanding differences in physical and virtual experiences for decision making (e.g., the choice to purchase a certain product). We completed this work across three iterations \todo{JRB: word-choice. Phases?} : an online survey, a low fidelity prototype, and a high-fidelity prototype.

\subsubsection{Online Survey}
We conducted a preliminary survey to identify important aspects of customers' in-store and online shopping experiences. We asked participants
% to incorporate, we first sent out a survey via social media.  We asked users 
to identify the three most important pieces of information involved their shopping decisions, what they did and did not like about existing in-store and online shopping experiences, and ways they currently use technology in shopping. \todo{does this feel like a reasonable synopsis?}  \todo{JRB: Yes, except I would like to know if these were open-ended questions or not.}
%We also gathered information about critical factors when deciding whether to purchase a product.  
%We gathered data on shoppers' propensity to use a phone or tablet when shopping and to research a product before making a purchase decision.
We used the responses from our online survey to isolate factors of in-store and online shopping that were most important to our participants. \todo{Can you add some details about how you analyzed this data here?}  \todo{JRB: Amen! Ethan and Matt, talk about this one? Was it a thematic analysis? Did you just see what people were saying? If you can describe what you did to me in casual english, I can help you science it up.}

\subsection{Low Fidelity Prototype}
We used the results from our preliminary survey to design a sketch-based low fidelity prototype of an AR system containing aspects of in-store and online shopping experiences that participants identified as important. Several such aspects for in-store decision making, such as being ``hands-on with the product'' and leaving the store with the product, are not readily feasible in an online experience; however, critical factors of online experiences, such as access to reviews and ease of comparison, can be brought into the store environment. \todo{JRB: Reword the previous sentence. Start by saying what you did do, and then what you could not.} We hypothesized that an augmented reality application that supplemented a traditional in-store experience with immediate access to core aspects of online shopping would improve consumer's confidence in their purchasing decisions. \todo{replace the last bit of this sentence with what was actually tested}

\todo[inline]{JRB: The paragraph below talks about one prototype. I think you should say that you had two. One prototype for interaction concept.}
We tested our hypothesis in a think-aloud study using our sketch-based prototype as a design prompt. \todo{JRB: Formative design work and evaluation do not have to be hypothesis driven, and often aren't. Instead they are exploratory. If you are feeling encumbered by the "hypothesis" language, we can reword. Just let me know.} The prototype consisted of online content drawn as AR menus on transparency sheets and overlaid onto an image of an electronics store. We used the low fidelity prototype to conduct a within-subjects A/B comparison of a menu-based, hierarchically structured user experience, and a context-aware virtual overlays of product information. \todo{any specific information? did you randomize presentation order?} \todo{JRB: The previous sentence sounds really smart. However, it doesn't really tell me what was happening. Simpler language might help: "Participants were interacted with each of the two prototypes, in a random order, and were allowed to navigate through a electronics retail experience....}  We instructed participants to use each prototype to aid in making a laptop purchase decision. \todo{what specific AR components were tested here (e.g., reviews, product comparisons, specs, etc.)? What were your measures here (e.g., confidence in the decision, time to decision, etc.)} \todo{JRB: How did you analyze the data from this phase?}
\todo[inline]{figure of paper prototype}

\subsection{High Fidelity Prototype}
We used the responses from the low fidelity prototype study to design a fully immersive StoAR prototype for use in head-mounted displays. Because we did not have access to a retail testing environment, we simulated a retail display in virtual reality based on configurations found in a local retail outlet. We then added an augmented reality interface to the simulation to provide XXX.\todo{briefly describe the important pieces informed by the low-fi study} While our use of a VR store simulation removes immediate access to real life product found in a real world store, the fully immersive environment allowed us to closely control the relationship between the prototype AR interfaces and simulated products. %and reduced the environmental complexity to allow participants to focus solely on the prototype implementation. 

We used a post-hoc survey to measure people's responses to the immersive prototype. Participants first freely navigated the virtual representation of a store augmented with static content containing information about each device. 
%Given space constraints, the track pad on one of the Vive's controllers was employed for users to navigate within the scene.
After navigating the scene, participants reported their perceptions of the prototype's usability, potential impact on decision-making, utility of individual design components, perceived trade-offs compared to existing technologies and potential limitations of the approach. They also provided feedback on additional applications where they envisioned using mixed reality experiences for decision making.  \todo{figure of the high fi prototype}
%The survey asked for specific product-domains to which our system could be applied (i.e. electronics, furniture, books).  
