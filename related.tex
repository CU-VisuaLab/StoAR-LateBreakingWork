\section{Related Work}

Previous work applying augmented reality to the shopping experience has focused on how designers and system architects properly utilize this novel technology \cite{ahn2015supporting,kourouthanassis2007enhancing,olsson2013expected,spreer2012improving,stoyanova2015comparison,zhu2004personalized}.  Context-awareness is a useful affordance provided by augmented reality.  Designers find that this ability to visually associate digital content with products increases customer empowerment, user efficiency in gathering information, and system influence. \todo{cite source?} Other AR application domains focus on collaborative experiences and visualizations \cite{esser2016head,santos2016augmented,truong2013today}.

Much of this previous work has focused on mobile augmented reality (MAR) as a display technology. \todo{explain what mobile augmented reality is - probably as simple as just saying "augmented reality using a mobile phone"}  However, with products such as Microsoft's Holo lens reaching the mainstream, head-mounted displays (HMD) are gaining attention as a potential platform for AR experiences.  Use of a dedicated HMD overcomes MAR-specific limitations such as insufficient processing power, required use of hands and an intermediate screen, and a field of view constrained by screen size \cite{bimber2005spatial}.  

Other work has been done to develop spatial augmented reality (SAR) systems and applications.  SAR applications employ ubiquitous computing to project context-aware digital content into the user's physical environment \cite{benko2015fovear,benko2014dyadic}.  Technical advantages of a SAR system include removing the need for users to wear or carry often cumbersome equipment and a non-restricted field of view.  Resolution could potentially improve, since computation and graphics processing hardware would not be constrained by form factor - a limitation for HMD which must not be too bulky to be worn on the user's head.  However, one limitation of SAR systems is that they require a static, controlled environment to be used effectively due to the use of fixed displays. \todo{is there a source that supports this? From your notes it seems that "Spatial Augmented Reality Merging Real and Virtual Worlds" might}

In this paper, we take a user-centered design approach to testing how the theoretical grounding derived from this previous work on mobile and spatial augmented reality applies to HMD-based augmented reality. We also examine how HMD-based AR may combine some of the benefits of these approaches while removing some of their limitations.