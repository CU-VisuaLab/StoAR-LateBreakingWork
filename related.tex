\section{Related Work}

Previous work applying augmented reality to the shopping experience has focused on how designers and system architects properly utilize this novel technology \cite{ahn2015supporting,kourouthanassis2007enhancing,olsson2013expected,spreer2012improving,stoyanova2015comparison,zhu2004personalized}.  Context-awareness is a useful affordance provided by augmented reality.  Designers find that this ability to visually associate digital content with products increases customer empowerment, user efficiency in gathering information, and system influence.  Other application domains additionally focus on collaborative experience and visualizations \cite{esser2016head,santos2016augmented,truong2013today}.

Much of this previous work has focused on mobile augmented reality (MAR) as a display technology.  However with products such as Microsoft's Holo lens, head-mounted displays (HMD) are gaining attention as a commercialized product.  Use of a dedicated HMD overcomes MAR-specific limitations of insufficient processing power, required use of hands and an intermediate screen, and a field of view constrained by screen size \cite{bimber2005spatial}.  In this paper, we take a user-centered design approach to test how the theoretical grounding derived from this previous work applies to HMD-based augmented reality.

Other work has been done to develop spatial augmented reality (SAR) systems and applications.  Spatial augmented reality employs ubiquitous computing to display context-aware digital content.  In systems such as those presented in \cite{benko2015fovear,benko2014dyadic}, projections augment the physical environment.  Technical advantages of a SAR system include user detachment from often cumbersome equipment and a non-restricted field of view.  Resolution could potentially improve, since computation and graphics processing hardware would not be constrained by form factor.  In contrast a head mounted display cannot have too bulky to be worn on a user's head.

When creating a spatial augmented reality system, the designer uses fixed displays.  This makes SAR a better candidate for a static, controlled environment than for a dynamic one.  Stores are mentioned as a potential application of IBM's Everywhere Displays technology \cite{pinhanez2001everywhere}.  We tested HMD-based augmented reality's ability to merge online and in-store shopping experiences, but as the technology improves, studies may be done to test the viability of a SAR shopping experience.