\section{Introduction}

%\todo[inline]{Illustrating the broader research problem}
Many real world tasks can also be accomplished using technology. \todo{Let's come up with a better opening sentence. = JRB--Yes, please! (No strong feelings on what that might be=DNS)} For example, we shop both in stores and online, converse both face-to-face and over video chat, and learn in both physical and virtual classrooms. However, virtual and physical experiences have unique affordances: shoppers in the real world can see and feel a product, while virtual shoppers can access more product information, such as reviews or offers from other retailers. Augmented reality technologies allow designers to craft applications that combine aspects of virtual and physical experiences. However, to effectively combine multiple existing experiences into a single application, designers must identify the critical aspects of parallel experiences and how to leverage their respective trade-offs. \todo{need to revise wording here}
%% JRB -- The middle of this paragraph is where it really takes off and you get to the heart of the problem -- unique affordances, but how do we blend?

%\todo[inline]{What did we look at}
In this work, we use retail shopping as a case study for understanding how to effectively craft mixed physical/virtual experiences from existing practices. The retail industry has shifted more of its commerce online.  Consumers have the option to do their shopping either through a physical store or the same store's online presence, and most consumers have experience with both methods \cite{nyt}. Retail shopping also aligns well with theoretical guidelines for effective AR applications. \todo{Matt, did you have references in mind for this?} \todo{JRB: Either way, it would be good to give us an example or two in an e.g.} \todo{from your lit review summary, kourouthanassis2007enhancing "several [AR] characteristics good for retail" and olsson2013expected "user expectations of MAR systems" might be good sources though i'm not sure if these count as theoretical guidelines -EH } While specific prior studies of online and in-store shopping have measured the importance of individual factors (e.g., price comparison \cite{karlsson2005price}, privacy \cite{miyazaki2001consumer}, and convenience \cite{bednarz2010perceptions}), these studies generally model these factors with respect to isolated effects on consumer decision making rather than how they might inform new technologies for supporting consumers in practice.  \todo{DNS made a few small changes to wording here and start of next paragraph, someone sanity check please!}
%We apply a user-centered design approach to understand aspects of online and in-store shopping that are important to consumers and how designers might leverage these aspects to generate new technologies for retail shopping.
%We instead apply a generative approach towards understanding how systems might effectively leverage key components of online and in-store shopping. 
%% JRB -- Great topic sentence! 

%  Retailers have adapted the online and in-store shopping experiences in ways that suit the needs of online and in-store shoppers respectively.  (TODO:  I THINK SOME OF THIS NEEDS LITERATURE BACKING; THIS IS BASED ON HUNCHES)  
%%DS--Tried to reframe it in a super generic way above. 

%This leaves customers with a new dilemma--one which forces them to choose between the benefits of one shopping experience over the other, while accepting that experience's limitations.

%A good deal of research has been devoted to defining characteristics of an application for which augmented reality is applicable.  A hybrid retail experience matches many of the theoretical grounds laid out by previous work.  StoAR applies augmented reality head-mounted displays in the widely relatable retail domain, allowing for more widespread use of novel head-mounted display technology.
%\todo[inline]{Our approach}
Instead of focusing on in-store and online shopping per se, we took a generative approach towards understanding how systems might effectively leverage key components of online and in-store shopping. 
%This work instead explores how designers might effectively derive new mixed reality applications from existing physical and virtual experiences, such as in-store and online shopping. \todo{this is a rough transition to me. I think the sentence needs to lead with the shopping instead of ending with it to maintain continuity with the previous paragraph, where we were talking specifically about shopping. -EH} \todo{JRB: Agreed. Maybe: "Instead of focusing on in-store and online shopping per se, we use these experience to explore how designers might effectively derive new mixed reality applications from existing physical and virtual experiences."}
% a set of preliminary design guidelines that capture important aspects of mixed reality shopping experiences. 
We applied \todo{JRB: I've given this note on all papers this week, but you need to be careful about tense. Different writing styles prefer past vs. present tense. Figure out from DS which should be used, and stick to it. DNS--Does this help? Tense is usually one of the last pieces I fix, so worth a skim through the rest of the paper to confirm that we're entirely in past} an iterative, user-centered design approach to deriving new design considerations for AR from existing parallel experiences. We first conducted a survey to understand perceived trade-offs in in-store and online experiences. \todo{JRB: What were the results of the survey? Did they inform the design?} We then designed two low fidelity prototypes based on these experiences and assessed their potential utility through a formative design evaluation using a think-aloud study \todo{JRB: Add citation, mostly because it is Clayton Lewis, and we love him.}. The results of the evaluation informed the design of StoAR, a head-mounted virtual reality simulation of a traditional retail environment augmented with critical information \todo{JRB: According to? DNS--"according to surveyed users"?} synthesized from conventional online retail experiences.\todo{JRB: Are you trying to say that the information was synthesixed from the survey data? DNS--I believe so, we should make that clear} Our analysis of user responses to the StoAR prototype \todo{JRB: You haven't said that you got user responses to this prototype... Add that in above?} provides preliminary design considerations for effectively blending online and in-store shopping experiences to facilitate consumer decision making and how a design-based approach might inspire novel blended interactions unique to immersive applications.
% \todo{need to revise the wording here} We embody the results of each design iteration in StoAR, a head-mounted augmented reality prototype designed for retail shopping experiences.

%\todo[inline]{What did we find}
\todo[inline]{The next paragraph needs reworking after the findings section is finished}
Our results illustrate concrete trade-offs between online and in-store shopping experiences and how those trade-offs might inform the design of a mixed reality application. We found that the benefits of each experience generally mirror the drawbacks of the other.  Specifically, we found XXX\todo{pithy version of findings here} and identify preliminary evidence of the utility of head-mounted displays for these applications. Through the iterative development of StoAR, we provide preliminary an empirical groundwork for how designers might effectively merge norms of physical and virtual experiences into a hybrid mixed reality application. 
% Through StoAR, we propose that augmented reality via a head-mounted display is an appropriate medium in delivering this hybrid experience.

% DS-Removed for space
%\todo[inline]{Write this :-)}
%\noindent\textbf{Contributions:} Sentence on main finding. We make the following specific contributions:
%\begin{itemize}
%	\item A UCD approach to crafting blended experiences for retail shopping
%   \item A simulation platform for evaluating AR experiences
%\end{itemize}
%A sentence on how this generalizes


