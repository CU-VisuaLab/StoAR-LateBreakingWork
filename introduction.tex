\section{Introduction}

The retail industry has shifted more of its commerce to the online realm.  Shoppers have the options do their shopping at in the store or to engage with that same store's online presence.  Retailers have adapted the online and in-store shopping experiences in ways that suit the needs of online and in-store shoppers respectively.  (TODO:  I THINK SOME OF THIS NEEDS LITERATURE BACKING; THIS IS BASED ON HUNCHES)  This leaves customers with a new dilemma--one which forces them to choose between the benefits of one shopping experience over the other, while accepting that experience's limitations.

We discuss the benefits and drawbacks of the online and in-store shopping experiences later in this paper, and find that the benefits of one experience tend to line up with the drawbacks of the other.  So with this work, we aim to merge previously parallel norms into a hybrid virtual and physical one which empowers users with important aspects of both.  Through StoAR, we propose that augmented reality via a head-mounted display is an appropriate medium in delivering this hybrid experience.

A good deal of research has been devoted to defining characteristics of an application for which augmented reality is applicable.  A hybrid retail experience matches many of the theoretical grounds laid out by previous work.  StoAR applies augmented reality head-mounted displays in the widely relatable retail domain, allowing for more widespread use of novel head-mounted display technology.
