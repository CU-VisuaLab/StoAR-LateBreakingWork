\section{Introduction}

%\todo[inline]{Illustrating the broader research problem}
Many real world tasks can be accomplished using technology. For example, we shop both in stores and online, converse both face-to-face and over video conference, and learn in both physical and virtual classrooms. \todo{not wedded to these examples} However, virtual and physical experiences have unique affordances: shoppers in the real world can see and feel a product, while virtual shoppers can access more product information, such as reviews. Mixed reality technologies allow designers to craft experiences that combine aspects of virtual and physical experiences. However, to effectively leverage these technologies, designers must identify the critical aspects of parallel virtual and physical experiences and how they might be combined. \todo{need to revise wording here}

%\todo[inline]{What did we look at}
In this work, we use retail shopping as a case study for understanding how to effectively craft mixed reality applications from existing physical and virtual experiences. The retail industry has shifted more of its commerce online.  Consumers have the option to do their shopping either through a physical store or the same store's online presence, and most consumers have experience with both methods of shopping---consumers made an estimated 51\% of their purchases online in 2015 \cite{}. \todo{cite http://www.wsj.com/articles/survey-shows-rapid-growth-in-online-shopping-1465358582} Retail shopping also aligns well with theorhetical guidelines for effective mixed reality applications. \todo{Matt, did you have references in mind for this?} While specific trade-offs of online and in-store shopping have been measured (e.g., price comparison \cite{karlsson2005price}, privacy \cite{miyazaki2001consumer}, and convenience \cite{bednarz2010perceptions}), these prior studies generally seek to understand individual factors in consumer decision making rather than novel, comprehensive experiences. We instead apply a generative approach towards understanding how systems might effectively leverage key components of online and in-store shopping. 

%  Retailers have adapted the online and in-store shopping experiences in ways that suit the needs of online and in-store shoppers respectively.  (TODO:  I THINK SOME OF THIS NEEDS LITERATURE BACKING; THIS IS BASED ON HUNCHES)  
%%DS--Tried to reframe it in a super generic way above. 

%This leaves customers with a new dilemma--one which forces them to choose between the benefits of one shopping experience over the other, while accepting that experience's limitations.

%A good deal of research has been devoted to defining characteristics of an application for which augmented reality is applicable.  A hybrid retail experience matches many of the theoretical grounds laid out by previous work.  StoAR applies augmented reality head-mounted displays in the widely relatable retail domain, allowing for more widespread use of novel head-mounted display technology.
%\todo[inline]{Our approach}
We aim to understand how designers might effectively derive new mixed reality applications from existing physical and virtual experiences, such as in-store and online shopping. 
% a set of preliminary design guidelines that capture important aspects of mixed reality shopping experiences. 
We take an iterative, user-centered design approach to constructing these guidelines from existing experiences by surveying user responses to a series of increaingly sophisticated design prompts prototyping a mixed reality shopping experience.  \todo{need to revise the wording here} We embody the results of each design iteration to create StoAR, a head-mounted augmented reality prototype designed for retail shopping experiences.

%\todo[inline]{What did we find}
Our results illustrate concrete trade-offs between online and in-store shopping experiences and how those trade-offs might inform the design of a mixed reality application. We found that the benefits of each experience generally mirror the drawbacks of the other.  Specifically, we found XXX\todo{pithy version of findings here} and identify preliminary evidence of the utility of head-mounted displays for these applications. Through iterative development of StoAR, we provide preliminary empirical groundwork for how designers might effectively merge parallel norms of physical and virtual experiences into a hybrid mixed reality application. 
% Through StoAR, we propose that augmented reality via a head-mounted display is an appropriate medium in delivering this hybrid experience.

\todo[inline]{Write this :-)}
\noindent\textbf{Contributions:} Sentence on main finding. We make the following specific contributions:
\begin{itemize}
	\item a bulleted list of specific contributions
\end{itemize}
A sentence on how this generalizes


