\documentclass{sigchi-ext}
% Please be sure that you have the dependencies (i.e., additional
% LaTeX packages) to compile this example.
\usepackage[T1]{fontenc}
\usepackage{textcomp}
\usepackage[scaled=.92]{helvet} % for proper fonts
\usepackage{graphicx} % for EPS use the graphics package instead
\usepackage{balance}  % for useful for balancing the last columns
\usepackage{booktabs} % for pretty table rules
\usepackage{ccicons}  % for Creative Commons citation icons
\usepackage{ragged2e} % for tighter hyphenation

% \usepackage{marginnote} \usepackage[shortlabels]{enumitem}
% \usepackage{paralist}

%% EXAMPLE BEGIN -- HOW TO OVERRIDE THE DEFAULT COPYRIGHT STRIP --
% \copyrightinfo{Permission to make digital or hard copies of all or
% part of this work for personal or classroom use is granted without
% fee provided that copies are not made or distributed for profit or
% commercial advantage and that copies bear this notice and the full
% citation on the first page. Copyrights for components of this work
% owned by others than ACM must be honored. Abstracting with credit is
% permitted. To copy otherwise, or republish, to post on servers or to
% redistribute to lists, requires prior specific permission and/or a
% fee. Request permissions from permissions@acm.org.\\
% {\emph{CHI'14}}, April 26--May 1, 2014, Toronto, Canada. \\
% Copyright \copyright~2014 ACM ISBN/14/04...\$15.00. \\
% DOI string from ACM form confirmation}
%% EXAMPLE END

\begin{document}
\title{SIGCHI Extended Abstracts Sample File: \underline{N}ote
  \underline{I}nitial \underline{C}aps}

\section{Introduction}

The retail industry has shifted more of its commerce to the online realm.  Shoppers have the options do their shopping at in the store or to engage with that same store's online presence.  Retailers have adapted the online and in-store shopping experiences in ways that suit the needs of online and in-store shoppers respectively.  (TODO:  I THINK SOME OF THIS NEEDS LITERATURE BACKING; THIS IS BASED ON HUNCHES)  This leaves customers with a new dilemma--one which forces them to choose between the benefits of one shopping experience over the other, while accepting that experience's limitations.

We discuss the benefits and drawbacks of the online and in-store shopping experiences later in this paper, and find that the benefits of one experience tend to line up with the drawbacks of the other.  So with this work, we aim to merge previously parallel norms into a hybrid virtual and physical one which empowers users with important aspects of both.  Through StoAR, we propose that augmented reality via a head-mounted display is an appropriate medium in delivering this hybrid experience.

A good deal of research has been devoted to defining characteristics of an application for which augmented reality is applicable.  A hybrid retail experience matches many of the theoretical grounds laid out by previous work.  StoAR applies augmented reality head-mounted displays in the widely relatable retail domain, allowing for more widespread use of novel head-mounted display technology.

\end{document}

%%% Local Variables:
%%% mode: latex
%%% TeX-master: t
%%% End:
