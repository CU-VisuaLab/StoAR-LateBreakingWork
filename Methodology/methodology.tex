\documentclass{sigchi-ext}
% Please be sure that you have the dependencies (i.e., additional
% LaTeX packages) to compile this example.
\usepackage[T1]{fontenc}
\usepackage{textcomp}
\usepackage[scaled=.92]{helvet} % for proper fonts
\usepackage{graphicx} % for EPS use the graphics package instead
\usepackage{balance}  % for useful for balancing the last columns
\usepackage{booktabs} % for pretty table rules
\usepackage{ccicons}  % for Creative Commons citation icons
\usepackage{ragged2e} % for tighter hyphenation

% \usepackage{marginnote} \usepackage[shortlabels]{enumitem}
% \usepackage{paralist}

%% EXAMPLE BEGIN -- HOW TO OVERRIDE THE DEFAULT COPYRIGHT STRIP --
% \copyrightinfo{Permission to make digital or hard copies of all or
% part of this work for personal or classroom use is granted without
% fee provided that copies are not made or distributed for profit or
% commercial advantage and that copies bear this notice and the full
% citation on the first page. Copyrights for components of this work
% owned by others than ACM must be honored. Abstracting with credit is
% permitted. To copy otherwise, or republish, to post on servers or to
% redistribute to lists, requires prior specific permission and/or a
% fee. Request permissions from permissions@acm.org.\\
% {\emph{CHI'14}}, April 26--May 1, 2014, Toronto, Canada. \\
% Copyright \copyright~2014 ACM ISBN/14/04...\$15.00. \\
% DOI string from ACM form confirmation}
%% EXAMPLE END

\begin{document}
\title{SIGCHI Extended Abstracts Sample File: \underline{N}ote
  \underline{I}nitial \underline{C}aps}

\section{Methodology}

\subsubsection{Initial Survey}

In order to identify important aspects of customers' in-store shopping experience to retain and important aspects of the on line experience to incorporate, we first sent out a survey via social media.  We asked users to identify ways that the on line and in-store experiences help and inhibit them.  We also gathered what factors are most important to them when making shopping decisions about a product.  We gathered data on shoppers' propensity to use a phone or tablet when shopping and to research a product before making a purchase decision.

\subsection{Low Fidelity Prototype}

In order to test the concept of an in-store shopping experience, drew content on transparency sheets and overlaid the sheets onto a picture of the user's expected view of a Best Buy store.  The low fidelity prototype was an A/B test of a menu-based, hierarchically structured user experience and a context-aware overlaying of content.  We instructed participants to use the prototyped system to aid in making a laptop purchase decision. TODO: ADD FIGURES OF PAPER PROTOTYPES.

\subsection{High Fidelity Prototype}

After eliciting feedback on the low fidelity prototype, we used Unity 3D game engine to mock out the intended scene and to create panels of product information, overlaid onto the scene.  Users were able to navigate virtual representation of a store with laptops on tables, digesting the static content provided by the panels.  The virtual reality scene was presented to users in an HTC Vive.  Given space constraints, the track pad on one of the Vive's controllers was employed for users to navigate within the scene.

After navigating the scene, users were prompted to fill out a survey to give feedback on the prototype's usability and feasibility as well as the potential for use of augmented reality for an in-store shopping experience.  The survey asked for specific product-domains to which our system could be applied (i.e. electronics, furniture, books).  The survey asked for opinions on how the platform enhances decision-making, useful features for an augmented reality shopping experience, perceived trade-offs compared to existing technologies and concerns about a system such as this one.

\end{document}

%%% Local Variables:
%%% mode: latex
%%% TeX-master: t
%%% End:
